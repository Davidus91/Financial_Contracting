When critically approaching this paper a natural question arises: what are the things currently not modelled that could effect incentives -- particularly misalignment of interests -- and thus have an impact on contracting. Starting from intuition and strengthened by reviewing the related literature we find that competition in research could be a relevant aspect to consider. So we take a look at what impact competition has on the optimality of the option contract and then derive a theoretical hypothesis using the author`s original framework. To connect this idea to the empirical part of the paper, we prompt a test of our proposition that can be carried out using the current dataset. For illustrative purposes some preliminary results are reported as well.
As a starting point, we take a look at the literature related to this question. Firstly -- to see whether the question is relevant at all -- we checked and confirmed that competition in research is actually often present in practice as several empirical papers analyse different aspects of it. Secondly, we looked more closely into those that say something about the change in likelihood of terminating research projects under more competitive circumstances. Rao (2014) finds evidence that firms respond to competitors: they are less likely to continue their research as the number of successful competitors grows.  Gunther McGrath (2004) hypothesizes and supports with data-analysis that pharmaceutical firms value an exit option for their investments into R\&D more in case of new technological areas with intense competition then otherwise. A point can be made that benefiting from related ongoing researches is also possible for example via patent-disclosure (Magazzini, 2009). Or focusing development on complementary products could also have an opposing beneficial effects. However, an empirical analysis by Rao (2014) looking for such benefits concludes that its existence is conditional on certain characteristics of the firms and markets.
As a next step, we move on to analyse the question in the current model-framework. In addition to the paper, we further assume that increasing competition can be captured through a ceteris paribus decrease in the expected value of narrow rights (N). The intuitive reasoning behind it is that since our lead product is yet unidentified (and so still can progress in several directions), a patent and/or launch by an other firm would not mean that our (future) N lose their value completely. However, as the particular disease market gets more crowded finding a niche is becoming more difficult and less profitable so the expected value of N ceteris paribus decreases as a result of competition. Also note, that no other variable is effected by competition in this setup, as R`s payoff is by construction independent of N.


In the original framework a ceteris paribus  decrease in the expected value of N would on one hand lead to an increase in the incentives of F to behave opportunistically and terminate even if R exerts $e_n$.  On the other hand this would also make F`s threat on terminating in case of $e_b$ more credible. Using the notation of the paper if competition is more intensive the above effects would translate into lower $\gamma$ and $\Delta$ respectively, which would overall cause a shift in the incentive compatibility constraint. [Picture of Lemma 2 equation]. This suggests that due to bigger competition -- captured by ceteris paribus lower expected N - the difference in the conditional payments becomes smaller. In severe cases -- when F benefits more in absolute terms from termination then from continuation after $e_n$ - i.e. $\Gamma$ becomes negative -- we would observe positive payment from F in case of termination $(p_t)$. Testability of these predictions however might be problematic as currently there is no data in the dataset on payments connected to options, and we are not confident that they can be obtained. Additionally, with a continuous dependent variable a different estimation approach would also be required.
In hope of finding a more easily testable prediction, we look at the condition that ensures the optimality of option contracts. Based on Lemma 3 it is evident that with $\Delta$ decreasing, the condition is more likely to hold as the proposed optimal option-contract becomes more profitable. The intuition behind this result is that as the outside option becomes more desirable for F, the threat for R that F will terminate will also be greater. Consequently, F needs to pay less rent to commit to this action, which makes the option-contract cheaper for him.
Based on this expected result a testable prediction present itself. Using the already existing disease dummies, a measure of competition can be constructed based on how many research contracts were signed connected to the same disease in a +/- 4-year interval   of any contract. So each contract would now have a variable describing how competitive is its particular research environment. Following our conclusions derived from the model, when dividing the dataset into low and highly competitive subsamples , we would expect a significantly larger probability of option contracts in the latter sample. Though we did not check for robustness or considered any other setups we were curious to see if our prediction is indeed supported by the data, so we ran the same FE and OLS model as the authors used on our two subsamples. Results indicate that estimated coefficients ($\beta_{FE}=0.20$, $\beta_{OLS}=0.16$) for the highly competitive subsample are significant and higher then the full sample ($\beta_{FE}=0.14$, $\beta_{OLS}=0.13$) and the low subsample ($\beta_{FE}=0.10$, $\beta_{OLS}=0.08$). The estimated likelihood for the low sample is substantially lower, however it is not significant. Clearly, the estimation approach needs more refinement, but at a first glance it looks promising.
