In this section we want to investigate and evaluate the contractual possibilities to mitigate the incentive conflict (\grqq project cross-subsidization\grqq) in research alliances and the possible resulting inefficiencies (underinvestment in the R\&D-sector from the social welfare standpoint) within the underlying framework of the paper, focusing on the role of renegotiation and institutional instruments, e.g. tax credit/subsidy in late stages of research. Thus we seek to answer the question: How to hedge against the distortions created by information asymmetries and move towards the first best solution?
As indicated by the theoretical result of the paper an option contract with termination rights and reversion of intellectual property to the pharmaceutical company (F) is pay-off maximizing whenever preferred to the unconditional contract (second best solution), moreover a similar result holds when the parties can not commit not to renegotiate . Although the specified option contract remains pay-off maximizing for the financing firm, the range under which it will be preferred to the unconditional is reduced. These findings are in line with the literature on this topic [1], which indicates that giving the lenders the right to seize the firm`s assets may enable the entrepreneur to credibly commit not to expropriate the lenders  and that the presence of renegotiation potentially deteriorates the agents incentive to exert effort .

As discussed in the current literature (Gompers and Lerner 1996, Castello 2013), on the one hand the degree of information asymmetry will typically affect the duration of contract between the parties: a higher degree of information asymmetry will more likely lead to a shorter duration of contracts. The contracting parties rely under such uncertain conditions on simple short term agreements and renegotiate more frequently in order to avoid problems arising from the misalignment of interests. On the other hand the exchange of highly specific assets is more likely to induce long-term contracts, which in combination with asymmetric information lead to a more frequent use of covenants, which are used as a substitute for the costs and efforts incurred by frequent renegotiations, in order to remedy the agency problems. Applied to the setting of research alliances delineated in the paper the assets exchanged between the research and the financing firm in situations where the exerted effort is not contractible seem by nature to be not specific (no specified lead product), thus the use of covenants (termination options) might be stronger related to the costs and effort associated to renegotiation rather than to the extent of contractibility. It is clear that such an analysis is beyond the scope of the paper since the inclusion of costs associated with renegotiation in the presented model will not only be theoretically difficult (\grqq Hold-up problem\grqq, bargaining power issues, etc.), from the empirical point of view finding evidence for such costs in the presented dataset seems rather unrealistic. Nevertheless, the above considerations might have an impact on the external validity of the empirical results of the paper, which will be discussed in more detail in the following section.

Finally we want to evaluate possible institutional policies, which may help to reduce the problem of underinvestment in the R\&D sector by using the theoretical results derived in the paper. Since technological innovation and economical growth are closely related underinvestment in research activities has a great impact not only on social welfare but also on growth opportunities created by technological advancements. The result in the Set Up with liquid research firms (Lemma 2`) indicates that the specified option contract  which implements $e_N$ maximizes the pay-off of the financing firm without assigning payments for continuation ($p_C = 0$), moreover it achieves the maximum joint pay-off for both sides even in the case of contractible research effort. Thus whenever it is socially optimal to implement $e_n$  an option contract can remedy the problem of cross-subsidization and opportunistic behavior without incurring an efficiency loss to the lender. This implies that alleviating the financial constraint of the research firm prevents \grqq credit rationing\grqq and allows the first best solution. A possible institutional policy to implement such an environment is to grant the financing firm a tax credit or subsidies on research expenditures conditional on continuation of the research alliances as compensation for the incurred costs, similar approaches on the research firms side such as governmental guarantees for entering research alliances with approved public biotechnology corporations (with certain reputation or certification) may yield similar results in counteracting the underinvestment problem. The equilibrium effectiveness of the mentioned institutional instruments though depends not solely on the interaction between the contracting parties but also on the market environment in which they are currently operating. Competition, substitution of knowledge and in general the interaction between firms can reduce the benefits of policies used to enhance innovation and limit their impact on social welfare (as indicated by findings of Finger 2008) and therefore have to be taken into account when evaluating these instruments. The importance of development and research in the presented case of biotechnology industry, especially considering the development of new/improved drugs for widely spread diseases, undeniably justifies the associated effort from the social welfare standpoint.


%References:
%Castello A.M. Mitigating incentive conflicts in inter-firm relationships: Evidence from long-term supply contracts, Journal of Accounting and Economics, 56(1), 19 – 39.
%Gompers, P. and J. Lerner. “Use of Covenants: An Empirical Analysis of Venture Partnership Agreements.” Journal of Law and Economics 39 (1996): 463-498.
%Finger, Stephen R., An Empirical Analysis of R&D Competition in the Chemicals Industry (November 12, 2008). Available at SSRN.
%[1] Tirole J., The Theory of Corporate Finance, Princeton University Press (2006)
%There reffering to Hart and Moore (1994) and Holmstrom (1979).






